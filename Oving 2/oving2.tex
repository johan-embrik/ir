\documentclass[]{article}
\usepackage[utf8]{inputenc}
\usepackage[norsk]{babel}

\begin{document}

\title{TDT4117 - Øving 2}
\author{Ole-Christer Selvik, Håkon Løvdal og Kristoffer Andreas Dalby}
\date{Oktober 2013}
\maketitle

\pagebreak

\include{oppgave1}

\section*{Oppgave 2 - Interpolated Precision}
\subsection*{Deloppgave a}
Interpolated precision


\pagebreak
\section*{Oppgave 3 - Relevance Feedback}
\subsection*{Deloppgave a}
Meningen med Relevance Feedback er å bruke data fra en tidligere spørring for å gi bedre presisjon av relevans på den neste spørringen. Det er vanlig å tilegne seg informasjon om den forrige spørringen var relevant for brukeren og bruke denne informsjonen. 
\subsubsection*{Query Expansion}
Query Expansion er en operasjon som gjerne brukes for å få flere relevante resultater på en gitt spørring. Det fungerer ved at logikken som tar imot spørringen kjører spørringen på flere forskjellige måter med forskjellige variasjoner av spørringen som ble motatt. Dette er svært vanlig å gjøre i søkemotorer. Endringene som er vanlig å gjøre på spørringen er for eksempel å endre ord i strengen til synonymer med samme betydning slik at man får et bredere spekter med resultater. Andre endringer kan være og rette skrivefeil og prøve forskjellige bøyninger av ordet.
\subsubsection*{Term Reweighting}
Term Reweighting går ut på å justere en spørring basert på resultatene den forrige spørringen fikk. Man justerer ikke selve stringen i spørringen, men man tar heller og veier alle termene i spørringen på nytt basert på forekomster av termene i resultatene på søket.
\subsubsection*{What seperates the two?}



\pagebreak

\end{document}

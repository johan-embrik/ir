\documentclass[]{article}
\usepackage[utf8]{inputenc}
\usepackage[norsk]{babel}

\begin{document}

\title{TDT4117 - Øving 2}
\author{Ole-Christer Selvik, Håkon Løvdal og Kristoffer Andreas Dalby}
\date{Oktober 2013}
\maketitle

\pagebreak

\section*{Oppgave 1}
\subsection*{Deloppgave a}

\noindent Først la oss si hovedformålet med språkmodellen: Ideen med språkmodellen er å rangerere dokumenter etter sannsynligheten for at dokumentet inneholder/vil kunne generere søkestrengen/spørringen. 


Dette fungerer ved at modellen setter en sannsynlighet for en gitt sekvens med ord (spørring) ved hjelp av sannsynlighetsfordeling. Gitt et dokument, vil vi sette opp en språkmodell for dette dokumentet. Dette vil si at hver term i dokumentet gis en vekt basert på sannsynligheten for at det søkes etter den termen i dokumentet. Videre, ved en samling med dokumenter, og en spørring Q, vil dokumentene i samlingen rangeres basert på sannsynligheten for at de vil generere termene i spørringen. Det finnes flere former for språkmodellen, men den vanligste er unigram-modellen. 

\subsubsection*{Fordeler med språkmodellen}

\begin{itemize}
    \item Effektiv og rask i IR-oppgaver.
    \item Enkel og lett å bruke.
    \item Stort sett gode resultater på spørringer
\end{itemize}

\subsubsection*{Ulemper med språkmodellen}

\begin{itemize}
    \item Enkle, unigram-modeller som vanskeliggjør å inkludere brukerens tilbakemeldinger eller preferanser.
    \item Vanskelig å velge gode sannsynligheter for termer.
    \item Forutsetter at dokumentet at søkestreng omhandler samme tema/er av samme type.
    \item Må benytte "smoothing-teknikker" istedenfor tf$_{idf}$-vekter.
\end{itemize}

\subsection*{Deloppgave b}

For å lettere å forstå hvor de ulike termvektene i dokumentet kommer ifra, setter vi opp en språkmodell for hvert dokument:

\noindent P(q \textbar  M$_{d}$) gir oss sannsynligheten for at spørringen q er i et enkelt dokument (ikke tatt hensyn til en samling med dokumenter).

\vspace{2mm}

q = \{NTNU, campus\}

\vspace{2mm}

M$_{1}$ = \{NTNU is a university in Trondheim\}

\vspace{2mm}

\hspace{3.2ex} = \{$\frac{1}{6}$, $\frac{1}{6}$, $\frac{1}{6}$, $\frac{1}{6}$, $\frac{1}{6}$, $\frac{1}{6}$\}

\vspace{2mm}

P(q \textbar  M$_{1}$) = $\frac{1}{6}$

\vspace{5mm}

M$_{2}$ = \{Gløshaugen is a Campus at NTNU, Øya is another campus.\}\footnotemark[1]

\vspace{2mm}

\hspace{3.2ex} = \{$\frac{1}{10}$, $\frac{1}{10}$, $\frac{1}{10}$, $\frac{1}{10}$, $\frac{1}{10}$, $\frac{1}{10}$, $\frac{1}{10}$, $\frac{1}{10}$, $\frac{1}{10}$, $\frac{1}{10}$\}

\vspace{2mm}

P(q \textbar M$_{2}$) = $\frac{1}{10}$

\vspace{5mm}

\noindent Videre setter vi opp språkmodell for samlingen:\footnotemark[2]
\noindent Dette vil si at vi regner ut sannsynligheten for at et dokumentet i samlingen vil generere strengen i spørringen. Med dette kan vi rangere dokumentene.

\vspace{2mm}

$\lambda$ = 0.5

\vspace{2mm}

P(q, d$_{1}$) = ((1-0.5)$\frac{2}{16}$) + (0.5($\frac{1}{6}$))  x  ((1-0.5)$\frac{2}{16}$) + (0.5($\frac{0}{6}$))

\vspace{2mm}

\hspace{8.4ex} = $\frac{7}{48}$  x  $\frac{1}{16}$   =   $\frac{7}{768}$  =  0.00911  

\vspace{4mm}

P(q, d$_{2}$) = ((1-0.5)$\frac{2}{16}$) + (0.5($\frac{1}{10}$))  x  ((1-0.5)$\frac{2}{16}$) + (0.5($\frac{2}{10}$))

\vspace{2mm}

\hspace{8.4ex} = $\frac{9}{80}$  x  $\frac{13}{80}$   =   $\frac{117}{6400}$  =  0.01823  

\vspace{2mm} 

\noindent Som vi ser av dette er d$_{1}$ \textless  d$_{2}$, som tilsier at d$_{2}$ vil rangeres som det beste dokumentet. 

\footnotetext[1]{Vi forutsetter at språkmodellen ikke tar hensyn til store og små bokstaver.}
\footnotetext[2]{Formelen vi benytter til å regne ut er gitt i oppgaveteksten til oppgave 1b.}
\pagebreak

\section*{Oppgave 2 - Interpolated Precision}
\subsection*{Deloppgave a}
Interpolated precision


\pagebreak
\section*{Oppgave 3 - Relevance Feedback}
\subsection*{Deloppgave a}
Meningen med Relevance Feedback er å bruke data fra en tidligere spørring for å gi bedre presisjon av relevans på den neste spørringen. Det er vanlig å tilegne seg informasjon om den forrige spørringen var relevant for brukeren og bruke denne informsjonen. 
\subsubsection*{Query Expansion}
Query Expansion er en operasjon som gjerne brukes for å få flere relevante resultater på en gitt spørring. Det fungerer ved at logikken som tar imot spørringen kjører spørringen på flere forskjellige måter med forskjellige variasjoner av spørringen som ble motatt. Dette er svært vanlig å gjøre i søkemotorer. Endringene som er vanlig å gjøre på spørringen er for eksempel å endre ord i strengen til synonymer med samme betydning slik at man får et bredere spekter med resultater. Andre endringer kan være og rette skrivefeil og prøve forskjellige bøyninger av ordet.
\subsubsection*{Term Reweighting}
Term Reweighting går ut på å justere en spørring basert på resultatene den forrige spørringen fikk. Man justerer ikke selve stringen i spørringen, men man tar heller og veier alle termene i spørringen på nytt basert på forekomster av termene i resultatene på søket.
\subsubsection*{What seperates the two?}



\pagebreak

\end{document}
